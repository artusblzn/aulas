\documentclass[11pt]{beamer}
\mode<presentation>
\let\Tiny=\tiny
\usetheme{CambridgeUS}
\usefonttheme{professionalfonts}
\usepackage[brazil]{babel}
\usepackage[utf8]{inputenc}
\usepackage{amsfonts}
\usepackage{amssymb}
\usepackage{amsmath}
\newtheorem{mydef}{Definição}
\newtheorem{myexample}{Exemplo}
\usepackage{listings}
\usepackage{fancyvrb}
\usepackage{enumerate}

\lstset{basicstyle=\scriptsize,language=Java, showstringspaces=false}

\newcommand\blfootnote[1]{%
  \begingroup
  \renewcommand\thefootnote{}\footnote{#1}%
  \addtocounter{footnote}{-1}%
  \endgroup
}

\title{Introdução aos Sistemas de Bancos de Dados}
\author{}
\date{}

\begin{document}

\begin{frame}[plain]
    \titlepage
\end{frame}

\section{Introdução}

\begin{frame}{Introdução aos Sistemas de Bancos de Dados}
    \begin{itemize}
        \item Dados é um tema antigo, porém bastante atual.
        \item A ascenção da Inteligência Artificial e dos métodos de análise de dados, mesmo que com resultados bastante discutíveis, é assunto central tanto no mundo acadêmico como no meio empresarial e nos governos do mundo todo.
    \end{itemize}
\end{frame}

\begin{frame}{Introdução aos Sistemas de Bancos de Dados}
    \begin{itemize}
        \item Livros como \textbf{Moneyball: The Art of Winning an Unfair Game}\footnote{Adaptado para o cinema.} tornaram-se as bíblias modernas.
        \item Um artigo de Allen Barra para o The Guardian\footnote{Moneyball: was the book that changed baseball built on a false premise? em 21 de abril de 2017.} faz uma crítica bastante severa ao livro e polemiza se a abordagem baseada em estatísticas apresentada realmente condiz com a realidade.
        \item A própria realidade mostrou que alguns pontos do livro não são verdades absolutas.
        \item O método Moneyball se opõe a \textit{sacrifice bunts} e \textit{sacrifice flys}\footnote{Jogadas do beisebol.}, mas o Arizona Diamondbacks chegou a World Series, coisa que o inventor do método nunca conseguiu, usando \textit{bunts}.
    \end{itemize}
\end{frame}

\begin{frame}{Introdução aos Sistemas de Bancos de Dados}
    \begin{itemize}
        \item Dizer que está tomando decisões baseadas em evidências é um grande negócio.
        \item Na prática, a tomada de decisões utilizando dados é muito mais complexa, cara e difícil que o apresentado.
        \item Há também o agravante que, mesmo que o modelo adotado seja correto, o resultado não seja o esperado, pois sempre há riscos e, dada a complexidade do mundo real, é impossível mapear todas as variáveis possíveis.
        \item Um termo bem sugestivo na estatística, e que mostra como a incerteza ocorre, é \textbf{esperança}.
    \end{itemize}
\end{frame}

\begin{frame}{Introdução aos Sistemas de Bancos de Dados}
    \begin{itemize}
        \item A análise de dados possui dois problemas de grande complexidade.
        \item O primeiro deles é estabelecer um modelo matemático correto.
        \item Livros de ciências dos dados colocam como requisito não apenas o conhecimento em computação ou engenharia, mas fundamentos sólidos em matemática e estatística.
    \end{itemize}
\end{frame}

\begin{frame}{Introdução aos Sistemas de Bancos de Dados}
    \begin{itemize}
        \item O segundo problema é os sistemas de bancos de dados.
        \item Modelos estatísticos necessitam de volumes massivos de dados.
        \item Coletar e organizar estes dados é um grande desafio e não deve ser tratado de maneira simplória.
        \item Até mesmo sistemas pequenos podem carregar imensa complexidade.
    \end{itemize}
\end{frame}

\begin{frame}{Introdução aos Sistemas de Bancos de Dados}
    \begin{itemize}
        \item Como dito, as bases de dados não são coisas novas. Pelo contrário, já existem a milênios.
        \item A República Romana já armazenava os dados dos seus cidadãos através do censo. Isto permitia classificá-los de acordo com suas posses.
    \end{itemize}
\end{frame}

\begin{frame}{Introdução aos Sistemas de Bancos de Dados}
    Antes de continuar a discussão, porém, é necessário definir o que é uma base de dados:
    \begin{mydef}
        Uma base de dados é uma coleção de dados relacionados que é projetada, construída e povoada a partir de dados logicamente coerentes entre si, com propósito específico e que representam algum aspecto do mundo.\footnote{ELMASRI E NAVATHE, 2011.}
    \end{mydef}
\end{frame}

\begin{frame}{Introdução aos Sistemas de Bancos de Dados}
    \begin{itemize}
        \item Apesar da definição, não é raro chamarmos por banco (ou base) de dados os Sistemas de Gerenciamento de Bancos de Dados (SGBDs).
        \item Um SGBD é um \textit{software} que fornece serviços para "definir, construir, manipular e compartilhar bases de dados com vários usuários e aplicações".\footnote{ELMASRI E NAVATHE, 2011.}
    \end{itemize}
\end{frame}

\begin{frame}{Introdução aos Sistemas de Bancos de Dados}
    \begin{itemize}
        \item Por \textbf{definir} tem-se a definição dos metadados.
        \item Metadados são as especificações de tipos, estruturas e restrições dos dados.
        \item Por exemplo, tenha-se um sistema acadêmico, a definição de quais dados dos alunos serão armazenados e que o nome do aluno deve ser uma cadeia de caracteres (\textit{string}) de até 300 caracteres são exemplos de metadados.
    \end{itemize}
\end{frame}

\begin{frame}{Introdução aos Sistemas de Bancos de Dados}
    \begin{itemize}
        \item A \textbf{construção} da base de dados é o armazenamento de dados segundo sua especificação em um ambiente controlado pelo SGBD.
        \item A \textbf{manipulação} da base de dados corresponde às operações CRUD (\textit{Create}, \textit{Retrieve}, \textit{Update} e \textit{Delete}).
        \item O \textbf{compartilhamento} da base de dados envolve permitir acesso de diversos usuário e aplicações à base concorrentemente.
    \end{itemize}
\end{frame}

\section{SGBD como camada de abstração}

\begin{frame}{Sistemas de Gerenciamento de Bancos de Dados}
    \begin{itemize}
        \item Do ponto de vista macro, os SGBDs fornecem uma camada de abstração entre as aplicações e as bases de dados.
        \item A aplicação não necessita conhecer como os dados estão armazenados em baixo nível, ela apenas solicita os dados aos SGBD.
        \item O projeto do banco de dados está desacoplado da aplicação, sendo necessário apenas o mapeamento entre os dados organizados no banco e as variáveis ou objetos da aplicação.
    \end{itemize}
\end{frame}

\begin{frame}{Sistemas de Gerenciamento de Bancos de Dados}
    \begin{itemize}
        \item O SGBD possui um catálogo de dados do banco.
        \item Este catálogo (metadados) é uma estrutura de dados que fornece informações sobre como os dados estão logicamente organizados.
        \item Como os SGBDs são agnósticos em relação à base de dados, o catálogo de dados é bastante genérico e flexível, permitindo a modelagem de diferentes base de dados, porém ainda submissas ao paradigma adotado pelo SGBD.
    \end{itemize}
\end{frame}

\subsection{Transações}

\begin{frame}{Transações}
    \begin{itemize}
        \item Outro conceito importante implementado pelos SGBDs modernos é o de transação.
        \item Uma transação é um conjunto de ações que envolve um ou mais acessos ao banco de dados, como a leitura ou modificação dos registros.
    \end{itemize}
\end{frame}

\begin{frame}{ACID}
    \begin{itemize}
        \item É desejável que os SGBDs garantam que suas transações atendam às propriedades ACID (\textit{Atomicity}, \textit{Consistency}, \textit{Isolation} e \textit{Durability}).
        \item Nem todos os SGBDs garantem as propriedades.
    \end{itemize}
\end{frame}

\begin{frame}{Atomicidade}
    \begin{itemize}
        \item Atomicidade é uma propriedade desejável para as transações.
        \item Uma transação atômica é uma transação que ou é executada por inteiro ou não é executada.
        \item Em outras palavras, a transação não deve ser executada "pela metade".
        \item Imagine, por exemplo, uma operação de transferência de dinheiro, onde apenas o ato de retirar de uma conta é executado, mas a inserção do valor em outra não.
    \end{itemize}
\end{frame}

\begin{frame}{Consistência}
    \begin{itemize}
        \item O SGBD também deve buscar manter o base de dados em estado consistente.
        \item Isto significa que os dados não devem assumir valores que violem as regras de negócio.
        \item Caso uma transação leve o banco de dados a um estado inconsistente, ela deve ser desfeita.
    \end{itemize}
\end{frame}

\begin{frame}{Isolamento}
    \begin{itemize}
        \item As transações são, principalmente em ambientes concorrentes (onde há múltiplos usuários requisitando e modificando dados da base), preferencialmente isoladas.
        \item Isto significa que se duas transações desejam modificar um determinado dado, uma deve ser executada apenas após a primeira terminar.
        \item Caso contrário, se a primeira transação falhar e tiver de ser desfeita, poderá ser impossível retornar ao estado inicial.
    \end{itemize}
\end{frame}

\begin{frame}{Durabilidade}
    \begin{itemize}
        \item Por fim, é esperado que toda transação tenha efeito persistente.
        \item Isto quer dizer que a aplicação deve esperar que em caso de sucesso, a transação é gravada em disco (ou outro meio de armazenamento persistente).
    \end{itemize}
\end{frame}

\subsection{Serviços do SGBD}

\begin{frame}{Serviços do SGBD}
    \begin{itemize}
        \item O SGBD, no entanto é mais do que uma simples camada de abstração.
        \item Ele é um servidor e, por isso, provê uma série de serviços.
    \end{itemize}
\end{frame}

\begin{frame}{Controle de redundância}
    \begin{itemize}
        \item O primeiro deles é o controle de redundância.
        \item Tenha-se uma universidade privadas e seus registros.
        \item Os setores acadêmico e financeiro devem possuir os dados de um mesmo estudante.
        \item Estes dados tendem a ser replicados (redundantes).
    \end{itemize}
\end{frame}

\begin{frame}{Controle de redundância}
    \begin{itemize}
        \item Sem um SGBD, qualquer modificação em um dado redundante levará a duplo esforço e abre a possibilidade para inconsistência de dados, ou seja, dados que deveriam ser idênticos, mas não são.
        \item Imagine que um aluno desta universidade se matricule em mais disciplinas.
        \item Se estes dados, alterados no setor acadêmico, não forem também modificados no setor financeiro, haverá inconsistência dos dados e o valor da mensalidade poderá não ser o correto.
    \end{itemize}
\end{frame}

\begin{frame}{Controle de redundância}
    \begin{itemize}
        \item Os SGBDs, especialmente os relacionais, permitem o projeto de bancos de dados normalizados, ou seja, sem redundâncias.
        \item Entretanto, em alguns casos, será necessário recorrer à desnormalização para melhoria na performance de consultas.
        \item Isto significa estabelecer uma redundância controlada.
        \item É esperado de um bom SGBD ferramentas para lidar com esta redundância.
    \end{itemize}
\end{frame}

\begin{frame}{Armazenamento persistente de objetos}
    \begin{itemize}
        \item Alguns SGBDs possuem suporte ao armazenamento de objetos.
        \item Apesar de ser uma característica comum aos SGBDs orientados a objetos, SGBDs relacionais já implementam técnicas e ferramentas que aproximam o paradigma relacional ao orientados a objeto, sendo conhecidos como SGBDs objeto-relacionais.
        \item Deste modo, reduz-se o problema de divergência de impedância causado pela diferença entre o paradigma orientado a objetos, presente na maior parte das aplicações, e o paradigma relacional, presente nos SGBDs mais usados.
    \end{itemize}
\end{frame}

\begin{frame}{Otimização de consultas}
    \begin{itemize}
        \item Os SGBDs permitem a otimização de consultas através de diversas técnicas.
        \item A primeira delas é a indexação através de tabelas \textit{hash} ou árvore.
        \item Outra técnica é o uso de \textit{buffers} ou \textit{caches} para armazenar registros e índices na memória, reduzindo o \textit{overhead} da busca em disco.
        \item Por último, os SGBDs também possuem otimizações de baixo nível da linguagem SQL.
    \end{itemize}
\end{frame}

\begin{frame}{\textit{Backup} e recuperação de falhas}
    \begin{itemize}
        \item Falhas de \textit{hardware} e \textit{software} são inevitáveis.
        \item É esperado de todo bom SGBD ferramentas para implementação de políticas de \textit{backup} e recuperação de falhas.
    \end{itemize}
\end{frame}

\section{Quando não usar SGBDs}

\begin{frame}{Quando não usar SGBDs}
    \begin{itemize}
        \item Haverá casos onde usar um SGBD não é recomendado.
        \item Estes são casos específicos.
        \item Principalmente em sistemas embarcados, onde há enormes restrições de espaço e performance, os SGBDs podem não ser bem vindos.
        \item Como tudo na computação, a decisão dependerá do contexto, mais especificamente dos requisitos não-funcionais.
    \end{itemize}
\end{frame}

\begin{frame}{Referências}
    \begin{itemize}
        \item Elmasri, Ramez e Navathe, Shamkant. Sistemas de banco de dados. 2011. 6ª ed. Pearson.
        \item Barra, Allen. Moneyball: was the book that changed baseball built on a false premise?. The Guardian de 21 de abril de 2017.
    \end{itemize}
\end{frame}

\end{document}