\documentclass[11pt]{beamer}
\mode<presentation>
\let\Tiny=\tiny
\usetheme{CambridgeUS}
\usefonttheme{professionalfonts}
\usepackage[brazil]{babel}
\usepackage[utf8]{inputenc}

\title{Personas e Cenários}
\author{}}
\date{}

\begin{document}

   \begin{frame}[plain]
        \titlepage
   \end{frame}

   \section{Introdução}

   \begin{frame}{Introdução}
      \begin{itemize}
         \item Algumas experiências bem sucedidas, como Facebook, criaram o mito que para se ter sucesso basta ter inspiração.
         \item \textit{Softwares} construídos somente com base na inspiração tendem fracassar miseravelmente.
         \item A inspiração, nestes casos, não atende às reais necessidades do usuário.
         \item E mesmo em casos onde \textit{softwares} feitos "apenas" com inspiração são bem sucedidos, é necessário entender como ele é utilizado e quais suas lacunas.
      \end{itemize}
   \end{frame}

   \begin{frame}{Introdução}
      Em termos materiais, há três fatores comuns que impulsionam a criação de produtos de \textit{software}:
      \begin{itemize}
         \item necessidades de negócio e de consumo não atendidas pelos produtos atuais;
         \item insatisfação com os produtos de \textit{software} atuais;
         \item mudanças na tecnologia.
       \end{itemize}
   \end{frame}

   \begin{frame}{Introdução}
      \begin{itemize}
         \item Diferentemente dos projetos de software, os produtos de \textit{software} não são resultado dos requisitos de um cliente específico, mas é interessante conhecer seus usuários e clientes potenciais.
         \item Há diversas técnicas formais para estudo desses usuários, algumas caras e até mesmo infactíveis.
         \item Outra abordagem é utilizar maneiras mais informais, porém que podem trazer grande valor às etapas iniciais de desenvolvimento do produto de \textit{software}.
         \item O objetivo é, a partir do conhecimento sobre estes usuários, elaborar funcionalidades e restrições.
      \end{itemize}
   \end{frame}

   \begin{frame}{Funcionalidades e restrições}
      \begin{itemize}
         \item Funcionalidades são o que o usuário pode fazer no \textit{software}.
         \item Na literatura as funcionalidades também recebem o nome de requisitos funcionais.
         \item O usuário postar uma foto é um exemplo de funcionalidade.
      \end{itemize}
   \end{frame}

   \begin{frame}{Funcionalidades e restrições}
      \begin{itemize}
         \item Restrições são limitações impostas sobre o \textit{software}.
         \item Na literatura também são conhecidas como requisitos não-funcionais.
         \item O \textit{software} ter de rodar nos sistemas operacionais Linux, Windows e MacOS é um exemplo de restrição.
      \end{itemize}
   \end{frame}

   \section{Personas}

   \begin{frame}{Personas}
      \begin{itemize}
         \item Como dito anteriormente, o primeiro ponto é saber quem vai usar o produto de \textit{software}.
         \item Muitas vezes conhece-se alguma pessoa que possui o perfil de usuário, porém nem todos da equipe de desenvolvimento podem ter essa mesma noção.
         \item Por isso, é interessante a criação de personas.
      \end{itemize}
   \end{frame}

   \begin{frame}{Personas}
      "Persona é uma representação dos usuários mais comuns, baseada em uma parcela de tarefas críticas." (Tomlin, 2018; tradução nossa)
   \end{frame}

   \begin{frame}{Personas}
      \begin{itemize}
         \item Personas são "usuários imaginários", como personagens de um livro de ficção inspirados em pessoas reais.
         \item A persona é utilizada para mapear as necessidades e como as pessoas utilizam a aplicação no seu dia-a-dia.
         \item Por exemplo, se você está desenvolvendo um \textit{software} para escritórios de advocacia, um advogado, um recepcionista, um assistente legal são bons exemplos de personas a serem criados.
      \end{itemize}
   \end{frame}

   \begin{frame}{Personas}
      Há quatro razões principais para o uso de personas:
      \begin{itemize}
         \item adicionar contexto aos dados de UX comportamental;
         \item permitir o projeto centrado no usuário;
         \item ajudar no recrutamento para teste de usabilidade;
         \item diminuir o *scope creep*.
      \end{itemize}
   \end{frame}

   \begin{frame}{Personas}
      \begin{itemize}
         \item Com o uso de personas, pode-se analisar os dados de UX comportamental para definir não apenas como cada persona utiliza a aplicação, mas qual a relevância desta persona no uso da aplicação.
      \end{itemize}   
   \end{frame}

   \begin{frame}{Personas}
      \begin{itemize}
         \item Tenha-se um caso onde dois projetos de \textit{interface} completamente diferentes estão em disputa. 
         \item A partir disto, pode-se escolher o projeto de \textit{interface} baseado em resultados mais sólidos.
         \item O uso de personas contextualiza e agrupa usuários, facilitando aos tomadores de decisão verem o que fazer.
      \end{itemize}
   \end{frame}

   \begin{frame}{Personas}
      \begin{itemize}
         \item As personas também servem para delinear o escopo do projeto.
         \item Elas servem para lembrar a todos os envolvidos quais são os pontos mais importantes.
         \item Não apenas reduz o chamado \textit{scope creep}\footnote{quando o escopo cresce demasiadamente com o desenvolvimento do projeto}, mas permite traçar o caminho para a construção de um produto viável mínimo (MVP).
      \end{itemize}
   \end{frame}

   \begin{frame}
      Tomlin coloca alguns pontos importantes a serem levados em conta para criação de personas:
      \begin{itemize}
         \item são baseadas em pesquisas de campo;
         \item identificam padrões comuns de comportamento;
         \item focam no agora e não no futuro potencial das coisas;
         \item incluem foto, nome e estória breve;
         \item descrevem um problema ou tarefa que uma pessoa está tentando resolver;
         \item boas personas incluem ambientes típicos e/ou dispositivos utilizados;
         \item incluem a familiaridade da pessoa com o domínio do problema;
         \item definem duas ou três tarefas principais que a persona precisa realizar.
      \end{itemize}
   \end{frame}

   \begin{frame}{Personas}
      Tomlin coloca como atributos comuns de uma persona\footnote{Par}:
      \begin{itemize}
         \item foto;
         \item tarefas críticas;
         \item cenário e \textit{background};
         \item dispositivos;
         \item \textit{expertise} do domínio;
         \item ambiente;         
       \end{itemize}
   \end{frame}

   \begin{frame}{Personas}
      "José, 40 anos, é professor de um Curso Técnico em Informática no interior do Brasil. Fez sua graduação em Engenharia da Computação no Rio de Janeiro.\\
      Alguns de seus estudantes são surdos, porém, como José não é fluente na língua de sinais e nem possui formação necessária, encontra grandes dificuldades em criar maneiras para que seus estudantes surdos possam estudar em casa.\\begin{align*}
      Ele precisa de um ambiente que seja possível colocar aulas com tradução para língua de sinais, além de outras atividades adaptadas."
      Tarefas críticas:
      1. Publicar vídeos com conteúdos adaptados ao público surdo.
      2. Publicar atividades adaptadas ao público surdo.
   \end{frame}

   \begin{frame}{Personas}
      \begin{itemize}
         \item Para criação de personas, Tomlin dá bastante ênfase à aquisição de informações.
         \item Ele sistematiza a investigação contextual de conduta.
      \end{itemize}
   \end{frame}

   \begin{frame}{Investigação contextual de conduta}
      "Investigação contextual de conduta é um método de pesquisa etnográfica para projeto centrado no usuário em que o pesquisador encontra os usuário no local e observa como eles interagem com sistemas no seu próprio ambiente." (Tomlin, 2018; tradução nossa)
   \end{frame}

   \begin{frame}{Investigação contextual de conduta}
      São etapas da investigação contextual de conduta:
      \begin{enumerate}
         \item preparação;
         \item encontrar as pessoas a serem observadas;
         \item sessão de observação;
         \item consolidação e análise dos dados coletados.
      \end{enumerate}
   \end{frame}

   \begin{frame}{Investigação contextual de conduta - preparação}
      A preparação envolve decidir:
      \begin{itemize}
         \item quem será observado;
         \item onde será observado;
         \item o que será observado;
         \item quais perguntas serão feitas.
      \end{itemize}
   \end{frame}

   \begin{frame}{Investigação contextual de conduta - encontrar pessoas}
      \begin{itemize}
         \item Para encontrar pessoas a serem observadas, em alguns casos, basta ir às ruas.
         \item Em outros casos, como juizes, empresários, médicos e outros profissionais que não estão comumente em público, mas em ambientes fechados, é necessário um esforço adicional.
         \item Pode-se utilizar contatos próprios e/ou recrutadores.
      \end{itemize}
   \end{frame}

   \begin{frame}{Investigação contextual de conduta - encontrar pessoas}
      Para convencer as pessoas a participar do estudo, deve-se evitar palavras como:
      \begin{itemize}
         \item pesquisa;
         \item observação;
         \item teste;
         \item estudo;
      \end{itemize}
      No lugar, deve-se dizer que busca-se "conhecer um pouco mais" o que a pessoa faz.
   \end{frame}

   \begin{frame}{Investigação contextual de conduta - encontrar pessoas}
      \begin{itemize}
         \item Deve-se ter muito cuidado com a privacidade e a anonimicidade do estudo.
         \item As pessoas que confiam que o que for dito ou observado durante o estudo não será compartilhado tendem a expor mais suas ideias.
      \end{itemize}
   \end{frame}

   \begin{frame}{Investigação contextual de conduta - sessão de observação}
      Tomlin sugere que a equipe de pesquisa siga alguns pontos:
      \begin{itemize}
         \item ser sempre cordial com os observados;
         \item calma ao ouvir/observar;
         \item prestar atenção até mesmo em expressões não verbais;
         \item perguntar o porquê das respostas ou ações;
         \item anotar ao máximo.
      \end{itemize}
   \end{frame}

   \begin{frame}{Investigação contextual de conduta - consolidação dos dados}
      \begin{itemize}
         \item a consolidação dos dados deve ser feita de maneira imediata para que não se percam detalhes.
         \item quanto mais distante a consolidação e análise de dados é feita, mais chances da equipe de pesquisa esquecer pontos que podem ser muito importantes.
      \end{itemize}
   \end{frame}

   \section{Cenários}

   \begin{frame}{Cenários}
      \begin{itemize}
         \item Sommerville recomenda o uso de cenários para descobrir as funcionalidades.
         \item O cenário é a narrativa que descreve uma dada situação em que um usuário (uma persona) está interagindo com o \textit{software}.
         \item Deve descrever o problema do usuário e como ele fará para resolvê-lo.
         \item Não é necessário descrever o sistema nos mínimos detalhes.
      \end{itemize}
   \end{frame}

   \begin{frame}{Cenários}
      Um cenário deve conter:
      \begin{itemize}
         \item uma breve afirmação sobre seu objetivo geral;
         \item referências à persona envolvida e suas motivações;
         \item informação sobre o que está envolvido na atividade;
         \item se apropriado, explicação dos problemas que não poderão ser resolvidos com o sistema e de como estes problemas poderão ser resolvidos.
      \end{itemize}
   \end{frame}

   \begin{frame}{Cenários}
      "José está ensinando o básico de \textit{hardware} para a turma e nela está um estudante surdo. José leva a turma para o laboratório de manutenção e demonstra as peças, uma a uma, com ajuda do intérprete de língua de sinais. Entretanto, o estudante surdo tem dificuldade em assimilar os novos sinais para as peças. Por isso, José precisa que o estudante tenha material para estudar em casa e fixar os novos termos.\\
      Então, José, com auxílio do intérprete, cria um vídeo mostrando as peças do computador e seus nomes em português e em LIBRAS. José autentica-se no Portal Mão Amiga e vai à tela de criar aula. Preenche os dados da aula e sobe o vídeo, além de outros materiais didáticos de apoio."
   \end{frame}

   \begin{frame}{Cenários}
      \begin{itemize}
         \item A escrita de um cenário deve começar a partir de uma ou mais personas criadas.
         \item Os cenários surgem a partir da imaginação sobre o que esta persona pode realizar com o \textit{software}.
         \item A escrita de cenários não possui uma fórmula exata e vai depender do produto de software e seus objetivos.
         \item Alguns cenários falarão mais de mecanismos e outros menos. Porém, o objetivo é que sejam claros o bastante para que até uma pessoa leiga entenda.
         \item Não são necessários cenários para todos os possíveis usos do software. Eles devem servir como auxílio e não como um ponto de lentidão.
      \end{itemize}
   \end{frame}

   \begin{frame}{Referências}
      \begin{itemize}
         \item Tomlin, W. Craig. UX Optimization: Combining Behavioral UX and Usability Testing Data to Optimize Websites. 1ed. 2018. APress.
         \item Sommerville, Ian. Engineering Software Products: An Introduction to Modern Software Engineering. 1ed. 2021. Pearson Education.
      \end{itemize}
   \end{frame}  
\end{document}