\documentclass[11pt]{beamer}
\mode<presentation>
\let\Tiny=\tiny
\usetheme{CambridgeUS}
\usefonttheme{professionalfonts}
\usepackage[brazil]{babel}
\usepackage[utf8]{inputenc}
\usepackage{ucs}
\usepackage{amsfonts}
\usepackage{amssymb}
\usepackage{amsmath}
\newtheorem{mydef}{Definição}
\newtheorem{myexample}{Exemplo}
\usepackage{pbox}
\usepackage{multirow}

\title{User stories}
\author{}
\date{}

\begin{document}

    \begin{frame}[plain]
        \titlepage
    \end{frame}

   \section{User Stories}

   \begin{frame}{\textit{User Stories}}
      \begin{itemize}
         \item Nos métodos ágeis, uma maneira bastante utilizada para descrever os requisitos são as \textit{user stories} (estórias de usuário).
         \item Uma \textit{user story} é uma descrição concisa de um requisito do ponto de vista do usuário.
         \item Ela deve ser simples o bastante para que possa ser escritas em um pequeno cartão.
         \item Não descreve tudo que há de ser feito, mas serve como lembrete do que deverá ser feito.
      \end{itemize}
   \end{frame}

   \begin{frame}{\textit{User Stories}}
      Uma estória de usuário é composta por três partes:
      \begin{itemize}
         \item cartão (descrição);
         \item conversa (detalhes);
         \item confirmação (testes).
      \end{itemize}
   \end{frame}

   \begin{frame}{\textit{User Stories} - Cartão}
      O cartão define três aspectos da \textit{user story}:
      \begin{itemize}
         \item quem? $\rightarrow$ interessado;
         \item o quê? $\rightarrow$ necessidade;
         \item por quê? $\rightarrow$ resultado.
      \end{itemize}
   \end{frame}

   \subsection{Cartão}

   \begin{frame}{\textit{User Stories} - Cartão}
      "Eu, enquanto professor, desejo postar aulas com vídeo e outros materiais didáticos."
\vspace{1cm}

      Eu, enquanto \textbf{$<$PAPEL$>$}, desejo \textbf{$<$fazer algo$>$}.
   \end{frame}

   \begin{frame}{\textit{User Stories} - Cartão}
      A estória de usuário também pode conter uma razão ou justificativa, como:
\vspace{1cm}
       
      "Eu, enquanto professor, desejo postar aulas com vídeo e outros materiais didáticos \textbf{para que meus estudantes surdos possam estudar em casa}."
\vspace{1cm}

      A justificativa em negrito não é obrigatória, mas ela pode ajudar os desenvolvedores em alguns contextos onde a funcionalidade não é tão clara.
   \end{frame}

   \begin{frame}{\textit{User Stories} - Cartão}
      \begin{itemize}
         \item A escrita a estória deve ser focada na solução e não no problema.
         \item A estória deve ser sempre do ponto de vista do usuário.
         \item O gerente de produto (ou cliente) deve estar atento a quem de fato é o usuário da funcionalidade e qual benefício este obterá.
      \end{itemize}
   \end{frame}

   \begin{frame}{\textit{User Stories} - Cartão}
      \begin{itemize}
         \item As estórias de usuário são colocadas no \textit{Product Backlog} (se você usar Scrum).
         \item Deve-se evitar estórias negativas, pois é difícil implementar o que o sistema não deve fazer.
      \end{itemize}
   \end{frame}

   \begin{frame}{\textit{User Stories} - Cartão}
       "Eu, enquanto usuário, não quero que o sistema grave e transmita minhas informações a servidores externos."
\vspace{1cm}
       
       É melhor reescreve-la de uma maneira positiva:
\vspace{.5cm}
       
       "Eu, enquanto usuário, desejo controlar a informação que é grava e transmitida a servidores externos para que eu garanta que minha informação pessoal não é compartilhada."
    \end{frame}

    \begin{frame}{\textit{User Stories} - Cartão}
       As estórias de usuário também podem ser usadas para descrever restrições do \textit{software}.
\vspace{1cm}       

       "Eu, enquanto cliente, desejo poder utilizar o sistema nos sistemas operacionais Windows, Linux, MacOS e FreeBSD."
\vspace{1cm}
       
       "Eu, enquanto estudante surdo, desejo poder navegar no sistema através de uma interface adaptada para que tenha mais facilidade no seu uso."
   \end{frame}

   \begin{frame}{\textit{User Stories} - Cartão}
      Segundo Mike Cohn, são boas práticas para escrita de \textit{user stories}:
      \begin{itemize}
         \item escrever primeiro estórias ligadas aos objetivos dos usuários;
         \item escrever em voz ativa;
         \item o cliente deve escrever a estória \footnote{sob supervisão, obviamente};
         \item não numerar os cartões;
         \item cartões são apenas um lembrete do que será discutido;
         \item dividir estórias grandes ou complexas;
      \end{itemize}
   \end{frame}

   \subsection{INVEST}

   \begin{frame}{\textit{User Stories} - INVEST}
      Bill Wake, famoso autor sobre métodos ágeis, criou o acrônimo INVEST para criação de boas \textit{user stories}:
      \begin{itemize}
         \item I - independente;
         \item N - negociável;
         \item V - valiosa;
         \item E - estimável;
         \item S - pequena (\textit{small});
         \item T - testável.
      \end{itemize}
   \end{frame}

   \begin{frame}{\textit{User Stories} - Independente}
      \begin{itemize}
         \item As \textit{user stories} devem ser independentes entre si.
         \item Se uma estória de alta prioridade depende de uma estória de baixa prioridade, então inverte-se a lógica de priorização, pois a estória menos valiosa deverá ser implementada primeiro.
      \end{itemize}
   \end{frame}

   \begin{frame}{\textit{User Stories} - Independente}
      Outro problema da dependência entre estórias de usuário é em relação a estimativas. Tenha-se as seguintes estórias:
      \begin{itemize}
         \item Eu, enquanto Cliente, desejo pagar pela compra usando um cartão Visa.
         \item Eu, enquanto Cliente, desejo pagar pela compra usando um cartão MasterCard.
         \item Eu, enquanto Cliente, desejo pagar pela compra usando um cartão American Express.
      \end{itemize}
   \end{frame}

   \begin{frame}{\textit{User Stories} - Independente}
      \begin{itemize}
         \item As três estórias são bastante similares e possuem complexidades quase idênticas.
         \item A primeira será implementada em, talvez, 3 dias.
         \item As demais, em 1 dia cada. 
         \item As estimativas para a segunda e terceira estórias não corresponderão à realidade.
      \end{itemize}
   \end{frame}

   \begin{frame}{\textit{User Stories} - Independente}
      Baseado em uma sugestão de Mike Cohn para problema similar, tem-se o seguinte reagrupamento das \textit{user stories}:
      \begin{itemize}
         \item Eu, enquanto Cliente, desejo pagar pela compra usando uma bandeira de cartão.
         \item Eu, enquanto Cliente, desejo pagar pela compra usando outras duas bandeiras de cartão.
      \end{itemize}
   \end{frame}

   \begin{frame}{\textit{User Stories} - Negociável}
      \begin{itemize}
         \item O cartão da estória de usuário não deve detalhar o requisito por completo.
         \item Ele serve apenas como um lembrete gentil do que deverá ser feito.
         \item No planejamento da iteração e em sua execução que devem ser discutidos os detalhes.
      \end{itemize}
   \end{frame}

   \begin{frame}{\textit{User Stories} - Negociável}
      Os cartões devem possuir apenas a estória de usuário e, no máximo, um par de notas que servirão de lembrete para as discussões como no exemplo abaixo:
      \begin{itemize}
         \item Estória: Eu, enquanto Cliente, desejo pagar pela compra usando cartão de crédito.
         \item Nota: serão aceitos cartão da bandeira Discover?
         \item Nota para UI: não colocar campo "bandeira", pois pode ser descoberto através dos primeiros números do cartão.
      \end{itemize}
   \end{frame}

   \begin{frame}{\textit{User Stories} - Valiosa}
      \begin{itemize}
         \item As estórias de usuário devem ter valor para os usuários ou clientes.
         \item Deve-se lembrar, que, do ponto de vista ágil e moderno, o importante é o produto a ser desenvolvido e o valor que ele agrega aos usuários e clientes.
      \end{itemize}
   \end{frame}

   \begin{frame}{\textit{User Stories} - Estimável}
      Estimar uma estória de usuário é reconhecer a complexidade da estória. Há três fontes principais para dificuldade, ou impossibilidade, de se estimar uma estória:
      \begin{itemize}
         \item os desenvolvedores não possuem conhecimento do domínio;
         \item os desenvolvedores não possuem conhecimento técnico;
         \item a estória é muito grande.
      \end{itemize}
      No primeiro caso, deve-se conversar com os criadores da estória.\\ 
      No segundo, pode-se realizar um \textit{spike}, onde aos desenvolvedores é dado uma janela de tempo para conhecerem sobre a tecnologia.\\ 
      Se a estória for muito grande, então é necessário dividi-la.
   \end{frame}

   \begin{frame}{\textit{User Stories} - Pequena (\textit{Small})}
      \begin{itemize}
         \item O tamanho de uma estória de usuário é um problema.
         \item A estória deve ser, a princípio, pequena.
         \item Entretanto, se muito pequena, pode levar a uma quantidade muito grande de requisitos.
      \end{itemize}
   \end{frame}

   \begin{frame}{\textit{User Stories} - Pequena (\textit{Small})}
      Uma estória grande é chamada de épico. Os épicos são geralmente em dois tipos:
      \begin{itemize}
         \item estória composta;
         \item estória complexa.
      \end{itemize}
   \end{frame}

   \begin{frame}{\textit{User Stories} - Pequena (\textit{Small})}
      A estória composta é um \textit{epic} (épico) que agrega diversas estórias de usuário. Por exemplo:\\

      "Eu, enquanto usuário, desejo autenticar-me no sistema através de credenciais externas."
\vspace{1cm}       

      Pode ser quebrado em:
      \begin{itemize}
         \item "Eu, enquanto usuário, desejo autenticar-me no sistema através das minhas credenciais no Google."
         \item "Eu, enquanto usuário, desejo autenticar-me no sistema através das minhas credenciais no Facebook."
         \item "Eu, enquanto usuário, desejo autenticar-me no sistema através das minhas credenciais na Apple."
         \item "Eu, enquanto usuário, desejo autenticar-me no sistema através das minhas credenciais no Microsoft."
      \end{itemize}
   \end{frame}

   \begin{frame}{\textit{User Stories} - Pequena (\textit{Small})}
      \begin{itemize}
         \item A estória complexa é um \textit{epic} com uma ou mais partes com grau elevado de complexidade.
         \item Isso ocorre quando o \textit{epic} deve estender algum algoritmo complexo ou criar um totalmente novo.
         \item Geralmente são fortes candidatos a \textit{spike}.
      \end{itemize}
   \end{frame}

   \begin{frame}{\textit{User Stories} - Testável}
      \begin{itemize}
         \item As estórias de usuário devem ser prefencialmente testáveis do ponto de vista de testes automatizados.
         \item Em alguns casos, especialmente em requisitos não funcionais, os testes não poderão ser automatizados.
         \item  Ainda assim, estes devem ser a ínfima minoria.
      \end{itemize}
   \end{frame}

   \subsection{Conversas}

   \begin{frame}{\textit{User Stories} - Conversas}
      \begin{itemize}
         \item As conversas entre time de desenvolvimento, gerente de produto (ou cliente) e demais interessados serve para negociar os detalhes das \textit{user stories}.
         \item Elas são necessárias porque os cartões em si não bastam para especificar os detalhes das estórias.
         \item São estabelecidos também os critérios de aceitação das estórias, ou seja, as regras de como a funcionalidade deve se comportar.
      \end{itemize}
   \end{frame}

   \subsection{Confirmação}

   \begin{frame}{\textit{User Stories} - Confirmação}
      \begin{itemize}
         \item A confirmação compreende os critérios e testes de aceitação.
         \item Os critérios de aceitação são similares às regras de negócio.
         \item Um critério de aceitação deve ser expresso por um enunciado pequeno e de fácil entendimento.
      \end{itemize}
   \end{frame}

   \begin{frame}{\textit{User Stories} - Confirmação}
      Para a estória: ``Eu, enquanto Comprador, quero utilizar meu cartão de crédito no pagamento das minhas compras.''\\
      Tem-se os seguintes critérios de aceitação (que poderão ficar no verso do cartão):
      \begin{enumerate}
         \item somente podemos aceitar cartões de crédito com bandeiras com que temos convênio.
         \item somente podemos aceitar cartões de crédito com data de expiração no futuro.
         \item somente podemos aceitar cartões de crédito com número e nome do dono válidos.
      \end{enumerate}
   \end{frame}

   \begin{frame}{\textit{User Stories} - Confirmação}
      \begin{itemize}
         \item Cada critério de aceitação pode possuir uma série de testes de aceitação.
         \item Um teste de aceitação serve para verificar se as saídas são corretas, ou seja, se, sob ponto de vista de negócios, a funcionalidade, de fato, realiza o que se propõe de acordo com as regras de negócio.
         \item Obviamente, nem todos os critérios de aceitação podem ter testes de aceitação como, por exemplo, ``o botão deve ser azul''.
      \end{itemize}
   \end{frame}

   \begin{frame}{\textit{User Stories} - Confirmação}
      Para o critério de aceitação: ``somente podemos aceitar cartões de crédito com bandeiras com que temos convênio.''\footnote{Exemplo adaptado de K21.}\\
      Tem-se os seguintes testes de aceitação:
      \begin{itemize}
         \item Comprador utiliza cartão de crédito Visa
           \begin{itemize}
              \item Aceitou = correto.
              \item Recusou = errado, deve ser corrigido!
           \end{itemize}
         \item Comprador de Livros utiliza cartão de crédito Amex
           \begin{itemize}
              \item Aceitou = errado, deve ser corrigido!
              \item Recusou = correto.
           \end{itemize}
      \end{itemize}
   \end{frame}

   \begin{frame}{Referências}
      \begin{itemize}
          \item Sommerville, Ian. Software Engineering - Global Edition. 10ed. 2016. Pearson Education.
          \item Sommerville, Ian. Engineering Software Products: An Introduction to Modern Software Engineering. 1ed. 2021. Pearson Education. 
          \item Cohn, Mike. User Stories Applied: For Agile Software Development. 1ed. 2004. Addison Wesley.
          \item K21 Global. Como é a user story? https://k21.global/pt/blog/como-e-a-user-story
      \end{itemize}
    \end{frame}

\end{document}