\documentclass[11pt]{beamer}
\mode<presentation>
\let\Tiny=\tiny
\usetheme{CambridgeUS}
\usefonttheme{professionalfonts}
\usepackage[brazil]{babel}
\usepackage[utf8]{inputenc}
\usepackage{ucs}
\usepackage{amsfonts}
\usepackage{amssymb}
\usepackage{amsmath}
\newtheorem{mydef}{Definição}
\newtheorem{myexample}{Exemplo}
\usepackage{pbox}
\usepackage{multirow}

\title{Estimativas ágeis}
\author{}
\date{}

\begin{document}

    \begin{frame}[plain]
        \titlepage
    \end{frame}

   \section{Introdução}

    \begin{frame}{Estimativas}
        \begin{itemize}
            \item Desde o momento que se faz o comprimisso em desenvolver um \textit{software} tem-se a pergunta: "Quando será terminado?"
            \item Até mesmo produtos de \textit{software} desenvolvidos continuamente por diversos anos precisam de datas.
        \end{itemize}
    \end{frame}

    \begin{frame}{Estimativas}
        \begin{itemize}
            \item O simples fato de supor quando um \textit{software}, ou parte dele, será entregue já é uma estimativa.
            \item Estimar nada mais é que uma previsão do futuro.
            \item Na quase totalidade das vezes, a previsão não será precisa.
        \end{itemize}
    \end{frame}

    \begin{frame}{Estimativas}
        \begin{itemize}
            \item O fato de ser impreciso não torna estimar inútil ou sem ciência alguma.
            \item Pelo contrário, todo grupo ou organização deve continuamente melhorar sua maneira de estimar, tanto utilizando experiências anteriores quanto aprimorando técnicas.
        \end{itemize}
    \end{frame}

    \begin{frame}{Estimativas}
        \begin{itemize}
            \item Apesar da literatura em Engenharia de \textit{Software} se preocupar bastante com métodos formais para estimativa, chegando até a vislumbrar o uso de Inteligência Artificial para fazê-lo, o que se tem são formas pouco estruturadas e que dependem diretamente da experiência dos envolvidos.
            \item Nos métodos ágeis isso não é diferente.
            \item O foco dos métodos ágeis nas conversas entre os envolvidos, no planejamento incremental e em estar sempre pronto para mudar os rumos implica muitas vezes em colocar em segundo plano certos formalismos. 
        \end{itemize}
    \end{frame}

    \begin{frame}{\textit{Story points}}
        \begin{itemize}
            \item É importante que a equipe de desenvolvimento seja a responsável por estimar os requisitos coletivamente.
            \item Mol\o{}kken-\O{}stvold e J\o{}rgensen (2004) mostram que grupos com quatro \textit{experts} tendem a fazer estimativas menos otimistas do que um único indivíduo.
            \item Um trabalho anterior de J\o{}rgensen já havia mostrado que quando a estimativa é feita pelos desenvolvedores tem mais acurácia que por equipe externa.
            \item Mais do que isso, pessoas com grande interesse na realização do projeto tendem a fornecer estimativas irreais para que os projetos sejam aprovados.
        \end{itemize}
    \end{frame}

    \begin{frame}{\textit{Story points}}
        Alsaadi e Saeedi (2022) colocam como desafios para estimar os requisitos\footnote{mais especificamente, \textit{user stories}}:
        \begin{itemize}
            \item inexperiência dos membros da equipe;
            \item pressão externa;
            \item influência da sequência de estimativa das \textit{stories}.
        \end{itemize}
        Na literatura há trabalhos (Grimstad e J\o{}rgensen, 2009; J\o{}rgensen e Halkjelsvik, 2020) que indicam que após estimar \textit{stories} pequenas, a equipe tende a subestimar as demais \textit{stories} e após estimar \textit{stories} grandes, a superestimar as demais. 
    \end{frame}

    \section{\textit{Story points}}

    \begin{frame}{\textit{Story points}}
        \begin{itemize}
            \item As \textit{user stories} são (ou deveriam ser) um monopólio para a análise de requisitos dentro dos métodos ágeis.
            \item Porém, tanto a natureza das \textit{user stories} quanto dos próprios métodos ágeis (iterativos incrementais) criam dificuldades para estimativas e planejamentos a longo prazo.
        \end{itemize}
    \end{frame}

    \begin{frame}{\textit{Story points}}
        \begin{itemize}
            \item O fato dos cartões serem simples lembretes e não definirem exatamente os requisitos, podem tornar, em um primeiro momento, nebulosas as formas de estimativa.
            \item O contexto, porém, faz toda diferença.
            \item Ao mesmo passo que perde-se uma dada "segurança" em estimar em relação ao que é feito nos métodos orientados a planos, ganha-se em reatividade a mudanças.
            \item De nada adianta ter a melhor das estimativas se as mudanças são inevitáveis.
            \item Pelo contrário, planejar e estimar consome muitos recursos.
        \end{itemize}
    \end{frame}

    \begin{frame}{\textit{Story points}}
        \begin{itemize}
            \item De nada adianta ter a melhor das estimativas se as mudanças são inevitáveis.
            \item Pelo contrário, planejar e estimar consome muitos recursos.
        \end{itemize}
    \end{frame}

    \begin{frame}{\textit{Story points}}
        \begin{itemize}
            \item O fato de tornar mais "nebulosas" as estimativas não quer dizer que os métodos ágeis ou as \textit{user stories} tornem as estimativas mais imprecisas.
            \item O que ocorrer é que o método de estimar deve mudar.
            \item Seria contraditório utilizar um paradigma para levantar, analisar e implementar os requisitos ao mesmo tempo que se usa uma forma de estimar mais arcaica. 
        \end{itemize}
    \end{frame}

    \begin{frame}{\textit{Story points}}
        \begin{itemize}
            \item Os métodos ágeis e as \textit{user stories} trazem a estimativa por \textit{story points}.
            \item \textit{Story points} são uma medida relativa.
            \item Na literatura, é possível ver os \textit{story points} representando tamanho (complexidade) ou esforço.
            \item Independente do que significarem os \textit{story points} para uma equipe, eles devem ser claros e estáveis.
        \end{itemize}
    \end{frame}

    \begin{frame}{\textit{Story points}}
        Tem-se exemplo de duas \textit{user stories} e suas estimativas em \textit{story points} para um sistema de \textit{marketplace}:
        \begin{itemize}
            \item Eu, enquanto Cliente, desejo pagar minhas compras com cartão de crédito. \textbf{SP: 3} 
            \item Eu, enquanto Logista, desejo cadastrar produtos para poder vendê-los. \textbf{SP: 2}
        \end{itemize}
    \end{frame}

    \begin{frame}{\textit{Story points}}
        \begin{itemize}
            \item O ponto central aqui é que deve haver uma triangulação entre as \textit{user stories}.
            \item As \textit{user stories} com mais pontos atribuídos devem ser as mais complexas (ou requiram mais esforço).
            \item A equipe deve garantir que seja mantida a devida proporção entre as \textit{user stories}.
        \end{itemize}
    \end{frame}

    \begin{frame}{\textit{Story points}}
        \begin{itemize}
            \item O ponto central aqui é que deve haver uma triangulação entre as \textit{user stories}, buscando que as \textit{user stories} com mais pontos atribuídos sejam mais complexas (ou requiram mais esforço).
            \item As faixas de pontuações variam de equipe para equipe, sendo a sequência de Fibonacci bastante utilizada.
            \item Deve-se tomar cuidado com valores muito altos porque podem significar \textit{stories} muito grandes e que deverão ser divididas.
        \end{itemize}
    \end{frame}

    \begin{frame}{\textit{Story points}}
        \begin{itemize}
            \item Entre complexidade e esforço há uma dicotomia que deve ser bem discutida entre a equipe.
            \item A complexidade refere-se ao quão difícil é implementar um dado requisito.
            \item Já esforço é o quanto de trabalho será exigido da equipe para implementar esse mesmo requisito.
        \end{itemize}
    \end{frame}

    \begin{frame}{\textit{Story points}}
        \begin{itemize}
            \item Os defensores da estimativa por complexidade usarão o argumento que o esforço para implementar algo dependerá da experiência da equipe.
            \item Isso faz sentido se observado que nos métodos ágeis o código é coletivo e que todos os desenvolvedores são iguais perante ao método.
            \item Além disso, os métodos ágeis incentivam desenvolvedores "polivalentes".
        \end{itemize}
    \end{frame}

    \begin{frame}{\textit{Story points}}
        \begin{itemize}
            \item O incentivo à "polivalência" é um ponto positivo dos métodos ágeis.
            \item Se os membros da equipe de desenvolvimentos são mais "completos" em seus conhecimentos, diminui-se o impacto do \textit{turnover}\footnote{\textit{Turnover é quando a empresa/equipe perde um funcionário/membro.}}
        \end{itemize}
    \end{frame}

    \begin{frame}{\textit{Story points}}
        Esta "polivalência" tem como pontos fortes:
        \begin{itemize}
            \item menor dependência de membros individuais da equipe;
            \item mais capacidade de criar soluções;
            \item maior tendência à equipe aprender novas tecnologias e atualizar-se constantemente.
        \end{itemize}
    \end{frame}

    \begin{frame}{\textit{Story points}}
        \begin{itemize}
            \item Estimar baseado em complexidade pode aumentar o erro na estimativa.
            \item Mike Cohn dá como exemplo, no seu artigo "\textit{Story Points Estimate Effort Not Just Complexity}", um projeto onde a equipe é formada por um cirurgião neurologista e uma criança.
            \item O projeto possui dois requisitos, onde um é uma cirurgia neurológica relativamente simples\footnote{Se é que isso é possível.} e colar X selos de carta.
            \item O primeiro é muito mais complexo que o segundo.
            \item Se a criança ficar responsável pela cirurgia a medida por complexidade poderá até estar de acordo com a realidade.
            \item Porém, se o número de selos for grande demais, mesmo que o cirurgião realize a cirurgia e a criança cole os selos, o tempo poderá ser o mesmo, criando um erro de estimativa.
        \end{itemize}
    \end{frame}

    \begin{frame}{\textit{Story points}}
        Entretanto, Mike Cohn esquece é que estimar por esforço possui alguns problemas:
        \begin{itemize}
            \item se a \textit{user story} é tão gigantesca e repetitiva como colar X selos, talvez ela deva ser quebrada em várias partes ou até mesmo reescrita;
            \item acopla cada \textit{story point} a uma quantidade de tempo;
            \item acopla cada \textit{user story} a um pequeno grupo de desenvolvedores dentro da equipe\footnote{Quando não a um único indivíduo.}.
            \item se à "pessoa certa" é dada a atividade, então não há sentido em estimar coletivamente os requisitos, pois se um é a "pessoa certa", então ele também tem a \textit{expertise} para estimar o requisito.
        \end{itemize}
    \end{frame}

    \begin{frame}{\textit{Story points}}
        \begin{itemize}
            \item Mike Cohn, porém, está correto em dizer que a complexidade não é o bastante para estimar um requisito.
            \item Fatores como risco e incerteza também devem ser levados em conta.
        \end{itemize}
    \end{frame}

    \section{Planning Poker\texttrademark}

    \begin{frame}{Planning Poker\texttrademark}
        \begin{itemize}
            \item O Planning Poker\texttrademark é uma técnica utilizada pelas equipes ágeis para a estimativa das \textit{user stories}.
            \item Ela é derivada da técnica Wideband Delphi.
        \end{itemize}
    \end{frame}

    \begin{frame}{Planning Poker\texttrademark}
        \begin{itemize}
            \item No Planning Poker\texttrademark, cada desenvolvedor recebe um baralho de cartas.
            \item Cada carta possui um valor representando uma quantidade de \textit{story points}.
            \item Uma \textit{user story} é lida em voz alta e os desenvolvedores podem fazer perguntas para esclarecer detalhes.
            \item Sanadas as dúvidas, todos os desenvolvedores escolhem, cada um, uma carta e a mostram a todos simultaneamente.
            \item Caso não haja consenso imediato (todos colocarem o mesmo valor), os desenvolvedores que usaram a maior e a menor carta argumentam o porquê o fizeram.
            \item O processo repete-se até que haja consenso. 
        \end{itemize}
    \end{frame}

    \section{Velocidade}

    \begin{frame}{Velocidade}
        \begin{itemize}
            \item Ao se usar \textit{story points} para estimar os requisitos, muda-se também a maneira como se vê a eficiência da equipe de desenvolvimento.
            \item Como as iterações possuem tempo fixo, a quantidade de \textit{story points} realizadas pela equipe em uma iteração é chamada \textbf{velocidade}.
        \end{itemize}
    \end{frame}

    \begin{frame}{Velocidade}
        \begin{itemize}
            \item Se uma equipe completa uma quantidade de \textit{user stories} que tenha suas complexidades somadas igual a 30 \textit{story points}, então a velocidade do time é igual a 30.
            \item Para a próxima iteração espera-se a equipe consiga repetir a mesma velocidade.
        \end{itemize}
    \end{frame}

    \begin{frame}{Velocidade}
        São fatores que podem afetar a velocidade da equipe:
        \begin{itemize}
            \item \textit{expertise} do domínio pela equipe;
            \item domínio das tecnologias pela equipe;
            \item coesão da equipe;
            \item reuniões;
            \item feriados;
            \item etc.
        \end{itemize}
    \end{frame}

    \begin{frame}{Velocidade}
        A velocidade leva em conta também que:
        \begin{itemize}
            \item não foram feitas horas extras;
            \item as estimativas das \textit{user stories} foram coerentes;
            \item as \textit{user stories} realizadas foram bem escritas e independentes.
        \end{itemize}
    \end{frame}

    \begin{frame}{Referências}
        \begin{itemize}
            \item Cohn, Mike. User Stories Applied: For Agile Software Development. 1ed. 2004. Addison Wesley.
            \item Cohn, Mike. Four Reasons Agile Teams Estimate Product Backlog Items. 2022. https://www.mountaingoatsoftware.com/blog/four-reasons-agile-teams-estimate-product-backlog-items
            \item Cohn, Mike. Story Points Estimate Effort Not Just Complexity. 2010. https://www.mountaingoatsoftware.com/blog/its-effort-not-complexity
            \item Cohn, Mike. Agile Estimating and Planning. 1ed. 2006. Prentice Hall.
            \item Alsaadi, Bashaer; Saeedi, Kawther. Data-driven Effort Estimation Techniques of Agile User Stories: a systematic literature review. 2022. Artificial Intelligence Review.
            \item Mol\o{}kken-\O{}stvold, Kjetil; J\o{}rgensen, Magne. Group Processes in Software Effort Estimation. 2004. Empirical Software Engineering.  
            \item Sutherland, Jeff. Story Points: Why are they better than hours? 2013. https://www.scruminc.com/story-points-why-are-they-better-than/
        \end{itemize}
    \end{frame}

\end{document}