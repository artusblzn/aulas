\documentclass[11pt]{beamer}
\mode<presentation>
\let\Tiny=\tiny
\usetheme{CambridgeUS}
\usefonttheme{professionalfonts}
\usepackage[brazil]{babel}
\usepackage[utf8]{inputenc}
\usepackage{amsfonts}
\usepackage{amssymb}
\usepackage{amsmath}
\newtheorem{mydef}{Definição}
\newtheorem{myexample}{Exemplo}

\title{Introdução à Engenharia de \textit{Software}}
\author{}
\date{}

\begin{document}

    \begin{frame}[plain]
        \titlepage
    \end{frame}

    \begin{frame}
      \frametitle{Sistema de informação}
      \begin{itemize}
        \item Os dados têm, como origem, diversas fontes.
        \item Além disso, o processamento de dados (transformação de dados em informação) é realizado por diversas partes que se relacionam em maior ou menor grau.
        \item A este conjunto de partes, dá-se o nome de sistema de informação.
      \end{itemize}
    \end{frame}

    \begin{frame}
      \frametitle{Sistema de informação}
      Um sistema de informação é responsável por:
      \begin{itemize}
        \item estruturar os dados;
        \item transformar os dados;
        \item distribuir a informação.
      \end{itemize}
    \end{frame}

    \begin{frame}
      \frametitle{Sistemas de informação}
      \begin{itemize}
        \item Os sistemas de informação são mais antigos que o computador.
        \item Então, o uso de sistemas de informação não está obrigatoriamente atrelado ao uso de \textit{softwares}.
      \end{itemize}
    \end{frame}

    \begin{frame}
      \frametitle{Engenharia de \textit{Software}}
      \begin{itemize}
        \item Produzir \textit{software} não é algo simples.
        \item O \textit{software}, no fim das contas, é a tradução de um processo de negócio realizado pelo ser humano para o computador.
      \end{itemize}
    \end{frame}
    
    \begin{frame}
      \frametitle{Engenharia de \textit{Software}}
      \begin{itemize}
        \item Para dificultar mais ainda, esses processos de negócio mudam constantemente devido a diversos fatores:
        \begin{itemize}
          \item políticos (internos e externos);
          \item econômicos;
          \item tecnológicos.
        \end{itemize}
      \end{itemize}
    \end{frame}
    
   \begin{frame}
      \frametitle{Engenharia de \textit{Software}}
      \begin{itemize}
        \item O ser humano não começou a desenvolver \textit{software} a tão pouco tempo.
        \item Pelo contrário, está a desenvolver \textit{software} há décadas e, entre acertos e falhas, descobriu processos e ferramentas que o auxiliam a diminuir as falhas no desenvolvimento de \textit{software}.
        \item A este arcabouço teórico e prático, dá-se o nome de \textbf{Engenharia de \textit{Software}}.
      \end{itemize}
   \end{frame}
        
   \begin{frame}
      \frametitle{Engenharia de \textit{Software}}
      \begin{itemize}
        \item A Engenharia de \textit{Software} foi inicialmente proposta em 1968 em uma Conferência.
        \item Até então, o desenvolvimento de \textit{software} era realizado de maneira bastante precária do ponto de vista da organização.
        \item Não havia um conjunto de boas práticas e organização.
        \item Isto levou à alta taxa de falha nos projetos de desenvolvimento de \textit{software}, além da entrega de produtos com qualidade aquém do esperado.
      \end{itemize}
   \end{frame}   
   
   \begin{frame}
      \frametitle{Engenharia de \textit{Software}}
      \begin{itemize}
        \item Para tentar resolver este problema, a Engenharia de \textit{Software} surge como uma disciplina que estuda, de maneira sistemática e organizada, os seguintes aspectos do desenvolvimento de \textit{software}:
        \begin{itemize}
           \item atividades;
           \item métodos;
           \item ferramentas;
           \item teorias;
           \item técnicas
        \end{itemize}
      \end{itemize}
   \end{frame}

   \begin{frame}
      \frametitle{Engenharia de \textit{Software}}
      \begin{itemize}
        \item Um ponto importante a ser salientado é que a Engenharia de \textit{Software} não estabelece uma resposta única para todos os problemas.
        \item Um \textit{software} para \textit{e-commerce} não deve ser desenvolvido da mesma maneira que um sistema embarcado para aviões.
        \item Apesar de haver características semelhantes (afinal, é tudo \textit{software}), a natureza de cada \textit{software}, seu contexto de uso, o tempo de desenvolvimento, seus custos associados, além de outros fatores são o que definirá as atividades, métodos, ferramentas e técnicas a serem utilizadas pela equipe de desenvolvimento.
        \item Em outras palavras, não há bala de prata.
      \end{itemize}
   \end{frame}
    
    \begin{frame}
      \frametitle{Processo de \textit{software} $\times$ Produto de software}
      \begin{itemize}
        \item Outro ponto a ser observado é se o processo de negócio já existe ou não.
        \item Nos primórdios do desenvolvimento de \textit{software}, os processos de negócio já existiam nas empresas e nos governos e eram traduzidos para \textit{software}.
        \item Este tipo de atividade gera dependência da maturidade e a definição do processo de negócio.
        \item Mais do que isso, gera a necessidade de um processo de desenvolvimento de \textit{software} bastante formal.
      \end{itemize}
    \end{frame}	    
    
    \begin{frame}
      \frametitle{Processo de \textit{software} $\times$ Produto de software}
      \begin{itemize}
        \item Por outro lado, vem ganhando força nas duas últimas décadas o foco no produto de \textit{software} e não no processo de \textit{software}.
        \item Isto quer dizer, a criação de \textit{software} é não atrelada obrigatoriamente a necessidades já existentes, mas a oportunidades.
        \item As redes sociais são um exemplo disso. Elas surgiram por observação de uma oportunidade e não por um processo de negócio já existente.
      \end{itemize}
    \end{frame}

    \begin{frame}
      \frametitle{Processo de \textit{software} $\times$ Produto de software}
      \begin{itemize}
        \item Entretanto, produto de \textit{software} não é algo novo ou revolucionário.
        \item Pelo contrário, \textit{softwares} de prateleira, como soluções para comércio ou \textit{softwares} de escritório, já existem há muito tempo.
        \item Entretanto, esses \textit{softwares} de prateleira e os softwares desta nova geração não podem ser colocados num mesmo grupo.
        \item Em aulas futuras veremos o que os diferencia tanto.
      \end{itemize}
    \end{frame}

    \begin{frame}{Referências}
      \begin{itemize}
          \item Sommerville, Ian. Software Engineering - Global Edition. 10ed. 2016. Pearson Education.
          \item Sommerville, Ian. Engineering Software Products: An Introduction to Modern Software Engineering. 1ed. 2021. Pearson Education. 
      \end{itemize}
   \end{frame}
\end{document}